\documentclass[
  man,
  floatsintext,
  longtable,
  nolmodern,
  notxfonts,
  notimes,
  colorlinks=true,linkcolor=blue,citecolor=blue,urlcolor=blue]{apa7}

\usepackage{amsmath}
\usepackage{amssymb}



\usepackage[bidi=default]{babel}
\babelprovide[main,import]{english}


% get rid of language-specific shorthands (see #6817):
\let\LanguageShortHands\languageshorthands
\def\languageshorthands#1{}

\RequirePackage{longtable}
\RequirePackage{threeparttablex}

\makeatletter
\renewcommand{\paragraph}{\@startsection{paragraph}{4}{\parindent}%
	{0\baselineskip \@plus 0.2ex \@minus 0.2ex}%
	{-.5em}%
	{\normalfont\normalsize\bfseries\typesectitle}}

\renewcommand{\subparagraph}[1]{\@startsection{subparagraph}{5}{0.5em}%
	{0\baselineskip \@plus 0.2ex \@minus 0.2ex}%
	{-\z@\relax}%
	{\normalfont\normalsize\bfseries\itshape\hspace{\parindent}{#1}\textit{\addperi}}{\relax}}
\makeatother




\usepackage{longtable, booktabs, multirow, multicol, colortbl, hhline, caption, array, float, xpatch}
\setcounter{topnumber}{2}
\setcounter{bottomnumber}{2}
\setcounter{totalnumber}{4}
\renewcommand{\topfraction}{0.85}
\renewcommand{\bottomfraction}{0.85}
\renewcommand{\textfraction}{0.15}
\renewcommand{\floatpagefraction}{0.7}

\usepackage{tcolorbox}
\tcbuselibrary{listings,theorems, breakable, skins}
\usepackage{fontawesome5}

\definecolor{quarto-callout-color}{HTML}{909090}
\definecolor{quarto-callout-note-color}{HTML}{0758E5}
\definecolor{quarto-callout-important-color}{HTML}{CC1914}
\definecolor{quarto-callout-warning-color}{HTML}{EB9113}
\definecolor{quarto-callout-tip-color}{HTML}{00A047}
\definecolor{quarto-callout-caution-color}{HTML}{FC5300}
\definecolor{quarto-callout-color-frame}{HTML}{ACACAC}
\definecolor{quarto-callout-note-color-frame}{HTML}{4582EC}
\definecolor{quarto-callout-important-color-frame}{HTML}{D9534F}
\definecolor{quarto-callout-warning-color-frame}{HTML}{F0AD4E}
\definecolor{quarto-callout-tip-color-frame}{HTML}{02B875}
\definecolor{quarto-callout-caution-color-frame}{HTML}{FD7E14}

%\newlength\Oldarrayrulewidth
%\newlength\Oldtabcolsep


\usepackage{hyperref}




\providecommand{\tightlist}{%
  \setlength{\itemsep}{0pt}\setlength{\parskip}{0pt}}
\usepackage{longtable,booktabs,array}
\usepackage{calc} % for calculating minipage widths
% Correct order of tables after \paragraph or \subparagraph
\usepackage{etoolbox}
\makeatletter
\patchcmd\longtable{\par}{\if@noskipsec\mbox{}\fi\par}{}{}
\makeatother
% Allow footnotes in longtable head/foot
\IfFileExists{footnotehyper.sty}{\usepackage{footnotehyper}}{\usepackage{footnote}}
\makesavenoteenv{longtable}

\usepackage{graphicx}
\makeatletter
\newsavebox\pandoc@box
\newcommand*\pandocbounded[1]{% scales image to fit in text height/width
  \sbox\pandoc@box{#1}%
  \Gscale@div\@tempa{\textheight}{\dimexpr\ht\pandoc@box+\dp\pandoc@box\relax}%
  \Gscale@div\@tempb{\linewidth}{\wd\pandoc@box}%
  \ifdim\@tempb\p@<\@tempa\p@\let\@tempa\@tempb\fi% select the smaller of both
  \ifdim\@tempa\p@<\p@\scalebox{\@tempa}{\usebox\pandoc@box}%
  \else\usebox{\pandoc@box}%
  \fi%
}
% Set default figure placement to htbp
\def\fps@figure{htbp}
\makeatother


% definitions for citeproc citations
\NewDocumentCommand\citeproctext{}{}
\NewDocumentCommand\citeproc{mm}{%
  \begingroup\def\citeproctext{#2}\cite{#1}\endgroup}
\makeatletter
 % allow citations to break across lines
 \let\@cite@ofmt\@firstofone
 % avoid brackets around text for \cite:
 \def\@biblabel#1{}
 \def\@cite#1#2{{#1\if@tempswa , #2\fi}}
\makeatother
\newlength{\cslhangindent}
\setlength{\cslhangindent}{1.5em}
\newlength{\csllabelwidth}
\setlength{\csllabelwidth}{3em}
\newenvironment{CSLReferences}[2] % #1 hanging-indent, #2 entry-spacing
 {\begin{list}{}{%
  \setlength{\itemindent}{0pt}
  \setlength{\leftmargin}{0pt}
  \setlength{\parsep}{0pt}
  % turn on hanging indent if param 1 is 1
  \ifodd #1
   \setlength{\leftmargin}{\cslhangindent}
   \setlength{\itemindent}{-1\cslhangindent}
  \fi
  % set entry spacing
  \setlength{\itemsep}{#2\baselineskip}}}
 {\end{list}}
\usepackage{calc}
\newcommand{\CSLBlock}[1]{\hfill\break\parbox[t]{\linewidth}{\strut\ignorespaces#1\strut}}
\newcommand{\CSLLeftMargin}[1]{\parbox[t]{\csllabelwidth}{\strut#1\strut}}
\newcommand{\CSLRightInline}[1]{\parbox[t]{\linewidth - \csllabelwidth}{\strut#1\strut}}
\newcommand{\CSLIndent}[1]{\hspace{\cslhangindent}#1}





\usepackage{newtx}

\defaultfontfeatures{Scale=MatchLowercase}
\defaultfontfeatures[\rmfamily]{Ligatures=TeX,Scale=1}





\title{Education Funding Inequality and Academic Performance Gap between
Migrant and Local Students in China}


\shorttitle{D2MR Final Project}


\usepackage{etoolbox}






\author{Jiayi Zou}



\affiliation{
{MA Program in the Social Sciences, University of Chicago}}




\leftheader{Zou}



\abstract{This document is a template demonstrating the apaquarto
format. It includes examples of how to create figures and tables, as
well as how to reference them in the text. The document is written in
Quarto, a system for creating documents with R Markdown. The apaquarto
extension provides a template for creating APA7-formatted manuscripts. }

\keywords{education inequality, internal migration, education
funding, fiscal decentralization}

\authornote{ 

\par{  This project is the final assignment for Data to Manuscript in R
(D2MR) instructed by Dr.~Natalie Dowling. It also serves as an interim
result of Jiayi Zou's MA thesis project.   The author is grateful
Dr.~Dowling for supporting this project and offering guidance throughout
the quarter.  }
\par{Correspondence concerning this article should be addressed to Jiayi
Zou, MA Program in the Social Sciences, University of Chicago, 1155 E
60th St., Chicago, IL 60637, USA, Email: jiayizou@uchicago.edu}
}

\makeatletter
\let\endoldlt\endlongtable
\def\endlongtable{
\hline
\endoldlt
}
\makeatother

\urlstyle{same}



\makeatletter
\@ifpackageloaded{caption}{}{\usepackage{caption}}
\AtBeginDocument{%
\ifdefined\contentsname
  \renewcommand*\contentsname{Table of contents}
\else
  \newcommand\contentsname{Table of contents}
\fi
\ifdefined\listfigurename
  \renewcommand*\listfigurename{List of Figures}
\else
  \newcommand\listfigurename{List of Figures}
\fi
\ifdefined\listtablename
  \renewcommand*\listtablename{List of Tables}
\else
  \newcommand\listtablename{List of Tables}
\fi
\ifdefined\figurename
  \renewcommand*\figurename{Figure}
\else
  \newcommand\figurename{Figure}
\fi
\ifdefined\tablename
  \renewcommand*\tablename{Table}
\else
  \newcommand\tablename{Table}
\fi
}
\@ifpackageloaded{float}{}{\usepackage{float}}
\floatstyle{ruled}
\@ifundefined{c@chapter}{\newfloat{codelisting}{h}{lop}}{\newfloat{codelisting}{h}{lop}[chapter]}
\floatname{codelisting}{Listing}
\newcommand*\listoflistings{\listof{codelisting}{List of Listings}}
\makeatother
\makeatletter
\makeatother
\makeatletter
\@ifpackageloaded{caption}{}{\usepackage{caption}}
\@ifpackageloaded{subcaption}{}{\usepackage{subcaption}}
\makeatother

% From https://tex.stackexchange.com/a/645996/211326
%%% apa7 doesn't want to add appendix section titles in the toc
%%% let's make it do it
\makeatletter
\xpatchcmd{\appendix}
  {\par}
  {\addcontentsline{toc}{section}{\@currentlabelname}\par}
  {}{}
\makeatother

%% Disable longtable counter
%% https://tex.stackexchange.com/a/248395/211326

\usepackage{etoolbox}

\makeatletter
\patchcmd{\LT@caption}
  {\bgroup}
  {\bgroup\global\LTpatch@captiontrue}
  {}{}
\patchcmd{\longtable}
  {\par}
  {\par\global\LTpatch@captionfalse}
  {}{}
\apptocmd{\endlongtable}
  {\ifLTpatch@caption\else\addtocounter{table}{-1}\fi}
  {}{}
\newif\ifLTpatch@caption
\makeatother

\begin{document}

\maketitle


\setcounter{secnumdepth}{-\maxdimen} % remove section numbering

\setlength\LTleft{0pt}


\section{Introduction}\label{introduction}

Internal migration in China has accelerated along with urbanization
since the implementation of Reform and Opening Up policy in the early
1980s. Statistics from the 7th National Census in 2020 show that over 70
million children in China have migration status, which means one fourth
of Chinese child population move interprovincially or intraprovincially
with their parents \footnote{See in
  \href{https://www.163.com/dy/article/JHFCU34705560ZWH.html}{Promoting
  reunion and avoiding separation - China's migrant children development
  report 2024}.}. Education and sociology research focusing on internal
migrant students found that these children have a relatively lower
school achievement compared to local students without migrant status,
and suffer from academic and financial difficulties, as well as
alienation in public education system
(\citeproc{ref-chenAccessPublicSchools2013}{Chen \& Feng, 2013};
\citeproc{ref-huangMythMigrantsProblems2017a}{Huang, 2017}).

Previous studies offered policy explanations for migrant students'
underachievement. Li (\citeproc{ref-liDataAnalysisCurrent2018}{2018})
indicated that central governments have less educational funding
distributed to provinces containing more migrant population due to
fiscal decentralization. On the other hand, the \emph{Hukou} policy
\footnote{The Hukou Policy is a population management policy that
  restraints non-local residents from/uplifts the threshold of enjoying
  the same social, medical, and educational public services as local
  households do.} has a history of limiting policy supports for internal
migrants including subsidies, fee standards, and other financial
accesses, which contributes migrant students' underperformance in school
(\citeproc{ref-luVillagersCityResilience2023}{Lu, 2023}).

However, both perspectives have failed to identify an integrated
framework: if we can discover the impact of differentiated financial
supports and per student funding appropriated to two groups of students,
then it is plausible to assume that fiscal decentralization is producing
local-migrant educational inequity through the lens of Hukou status.

In this study, I seek to understand \textbf{how governments'
differentiated provision of education fundings affects the academic
performance gap between migrant students and local students in China} .
My hypothesis is \textbf{when the government provides migrant students
with limited fundings, and less-supportive charging standards and
subsidy policies, the academic performance gap between the two groups of
students is likely to widen}.

Beyond measuring educational disparities created by the complexity of
fiscal decentralization and population management system, this research
has practical significance for addressing the ongoing migrant problems
in China's urban governance and the institution of compulsory education
(\citeproc{ref-nationalbureauofstatisticsofchinaetal_2023_what}{National
Bureau of Statistics of China et al., 2023}). Last but not least, this
study can also offer indications for how educational finance and
policies provided by government interacts with structural inequality in
other social contexts (e.g., areas with higher poverty level or racial
disparities, see in studies by Baird
(\citeproc{ref-baird_2008_federal}{2008}) and Hyman
(\citeproc{ref-hyman_2017_does}{2017})).

\section{Literature Review}\label{literature-review}

The imbalance of education finance provision increases the overall
stress of providing equal education service to local-registered
residents and migrant influx
(\citeproc{ref-jinRegionalDecentralizationFiscal2005a}{Jin et al.,
2005}). producing biased human capital investments in education system
(\citeproc{ref-heckmanChinasHumanCapital2005}{Heckman, 2005})

\section{Methods}\label{methods}

\section{Analysis Results}\label{analysis-results}

\section{Conclusion}\label{conclusion}

\clearpage

\section{References}\label{references}

\phantomsection\label{refs}
\begin{CSLReferences}{1}{0}
\bibitem[\citeproctext]{ref-baird_2008_federal}
Baird, K. E. (2008). Federal {Direct Expenditures} and {School Funding
Disparities}, 1990-2000. \emph{Journal of Education Finance},
\emph{33}(3), 297--310. \url{https://www.jstor.org/stable/40704331}

\bibitem[\citeproctext]{ref-chenAccessPublicSchools2013}
Chen, Y., \& Feng, S. (2013). Access to public schools and the education
of migrant children in {China}. \emph{China Economic Review}, \emph{26},
75--88. \url{https://doi.org/10.1016/j.chieco.2013.04.007}

\bibitem[\citeproctext]{ref-heckmanChinasHumanCapital2005}
Heckman, J. J. (2005). China's human capital investment. \emph{China
Economic Review}, \emph{16}(1), 50--70.
\url{https://doi.org/10.1016/j.chieco.2004.06.012}

\bibitem[\citeproctext]{ref-huangMythMigrantsProblems2017a}
Huang, E. (2017). The myth of 'migrants as problems': {Public} school in
neoliberal times and the construction and contestation of 'migrant'
identity. \emph{Journal of Language, Identity, and Education},
\emph{16}(6), 381--394.
\url{https://doi.org/10.1080/15348458.2017.1381022}

\bibitem[\citeproctext]{ref-hyman_2017_does}
Hyman, J. (2017). Does {Money Matter} in the {Long Run}? {Effects} of
{School Spending} on {Educational Attainment}. \emph{American Economic
Journal: Economic Policy}, \emph{9}(4), 256--280.
\url{https://www.jstor.org/stable/26598353}

\bibitem[\citeproctext]{ref-jinRegionalDecentralizationFiscal2005a}
Jin, H., Qian, Y., \& Weingast, B. R. (2005). Regional decentralization
and fiscal incentives: {Federalism}, {Chinese} style. \emph{Journal of
Public Economics}, \emph{89}(9-10), 1719--1742.
\url{https://doi.org/10.1016/j.jpubeco.2004.11.008}

\bibitem[\citeproctext]{ref-liDataAnalysisCurrent2018}
Li, N. (2018). \emph{{Data analysis: Current situation, problems and
countermeasures of the financial system of compulsory education for
migrant children}}.
https://www.thepaper.cn/newsDetail\_forward\_2673802.

\bibitem[\citeproctext]{ref-luVillagersCityResilience2023}
Lu, S. (2023). '{Villagers} in the {City}': {Resilience} in migrant
youth amidst urbanisation. \emph{British Journal of Social Work},
\emph{53}(1), 236--257. \url{https://doi.org/10.1093/bjsw/bcac122}

\bibitem[\citeproctext]{ref-nationalbureauofstatisticsofchinaetal_2023_what}
National Bureau of Statistics of China, UNICEF China, \& UNFPA China.
(2023). \emph{What the 2020 {Census Can Tell Us About Children} in
{China}: {Facts} and {Figures}}.
https://www.unicef.cn/en/reports/population-status-children-china-2020-census.

\end{CSLReferences}






\end{document}
